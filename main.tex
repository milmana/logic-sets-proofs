\documentclass{article}
\usepackage{amsmath}
\title{Proofs and Fundamentals}
\author{Arsenio M}
\begin{document}

\maketitle

\tableofcontents

\section{Informal Logic}

Law of Excluded Middle: every proposition is either true or false, with no third possibility.

\[ P \lor \neg P \]

Negatiion (NOT):

\[
	\begin{array}{c | c}
		P & \neg P \\ \hline
		\text{T} & \text{F} \\
		\text{F} & \text{T} \\
	\end{array}
\]

Conjunction (AND):

\[
	\begin{array}{c c | c}
		P & Q & P \land Q \\ \hline
		\text{T} & \text{T} & \text{T} \\
		\text{T} & \text{F} & \text{F} \\
		\text{F} & \text{T} & \text{F} \\
		\text{F} & \text{F} & \text{F} \\
	\end{array}
\]

Disjunction (inclusive OR):

\[
	\begin{array}{c c | c}
		P & Q & P \lor Q \\ \hline
		\text{T} & \text{T} & \text{T} \\
		\text{T} & \text{F} & \text{T} \\
		\text{F} & \text{T} & \text{T} \\
		\text{F} & \text{F} & \text{F} \\
	\end{array}
\]

Conditional (IF-THEN):

\[
	\begin{array}{c c | c}
		P & Q & P \rightarrow Q \\ \hline
		\text{T} & \text{T} & \text{T} \\
		\text{T} & \text{F} & \text{F} \\
		\text{F} & \text{T} & \text{T} \\
		\text{F} & \text{F} & \text{T} \\
	\end{array}
\]

Bi-conditional (???):

\[
	\begin{array}{c c | c}
		P & Q & P \leftrightarrow Q \\ \hline
		\text{T} & \text{T} & \text{T} \\
		\text{T} & \text{F} & \text{F} \\
		\text{F} & \text{T} & \text{F} \\
		\text{F} & \text{F} & \text{T} \\
	\end{array}
\]

A tautology is a statement that is always true by logical necessity, regardless of whether the component statements are true or false, and regardless of what we happen to observe in the real world.

\[
	\begin{array}{c c | c}
		P & \neg P & P \lor \neg P \\ \hline
		\text{T} & \text{F} & \text{T} \\
		\text{F} & \text{T} & \text{T} \\
	\end{array}
\]

A contradiction is a statement that is always false by logical necessity.

\[
	\begin{array}{c c | c}
		P & \neg P & P \land \neg P \\ \hline
		\text{T} & \text{F} & \text{F} \\
		\text{F} & \text{T} & \text{F} \\
	\end{array}
\]

Implication vs equivalence:
\begin{itemize}
	\item Implication $\Rightarrow$ is conditional $\rightarrow$ that is a \textbf{tautology}.
	\item Equivalence $\Leftrightarrow$ is biconditional $\leftrightarrow$ that is a \textbf{contradiction}.
\end{itemize}

Sentential (propositional) logic vs first-order (predicate) logic:
\begin{itemize}
	\item Universal quantifier $\forall$,
	\item Existential quantifier $\exists$.
\end{itemize}



\end{document}
